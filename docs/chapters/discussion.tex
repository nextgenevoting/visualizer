\chapter{Conclusion}

The future of e-voting in Switzerland is still unclear. With the efforts made
to publish the CHvote specification, finally a lot of issues involved with
e-voting have been solved. Unfortunately, an ordinary citizen does not have any
in-depth knowledge in cryptography in order to understand a protocol like
CHvote. So the introduction of a modern e-voting system in Switzerland or in
any other country in general depends on how strongly people believe and trust
such a protocol. This is were we hope that our application will help people to
understand how a modern e-voting system works. Although the protocol itself is
quite complicated, our application allows to explain and visualize the basic
ideas of modern e-voting.

\section{Reflection}

Vue.js turned out to be the perfect framework for coding the front-end of the
application. It allowed us to develop in small bits, always being able to
directly preview each step rendered in the browser. As for the back-end, we
still strongly believe that the choice of Python wasn't the correct one for
this kind of project. Python seems to be the number one choice for prototyping,
but for larger applications a lot of issues appear while developing.
Unfortunately, as we already implemented the CHvote protocol as part of
"Projekt 2" in Python, it did not make much sense to rewrite everything in a
different language, since it would have been too time consuming.


