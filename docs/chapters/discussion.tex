\chapter{Conclusion}

During the first few weeks we felt as if we have been thrown into cold water. Reading and understanding the protocol wasn't easy at first, because we had to get used to the notation and memorize a large amount of variables used by the many algorithms. While some of the cryptographic primitives were taught in previous courses, most of them were new and unknown to us. We focused on getting a good understanding of the protocol on a higher level rather than learning about each and every algorithm in detail, as this was sufficient for implementing and understanding the protocol.

Additionally, programming algorithms isn't something we are doing on a daily basis. Therefore, the first few algorithms took us quite some time to implement. After a few weeks, we could greatly increase our productivity and in the end, we could implement even the larger algorithms in not much more time than the simple ones in the beginning of the project.

From our perspective, the project has been extremely interesting and we are still impressed by the ideas presented and specified in the CHVote specification. From simply implementing the protocol we could learn a lot about the CHVote protocol and E-Voting in general and could improve both our knowledge of more advanced cryptographic topics and get practise in implementing cryptographic algorithms.

\section{Python drawbacks}

During the project we have experienced a few issues with the programming language that we used to implement the specification in, Python. In particular, we have observed the following issues:

\begin{itemize}
	\item Performance issues due to Python being an interpreted language
	\item Function overhead: function calls in Python seem to be quite slow
	\item Strongly dynamic typing vs. static typing: the Python interpreter needs to inspect every single object during run time (be it an integer or a more complex object)
	\item The \textit{BigInteger} library surprisingly isn't as fast as using directly the GMP library
	\item Larger projects tend to turn out messy
	\item Little to no standard documentation regarding project structure
	\item No real standard for unit testing, documentation generation etc.
\end{itemize}

For detailed information regarding the performance issues that we have experienced see \cite{slowpy} and \cite{slowpy2}. Based on the reasons above we would not recommend to use Python for the use in similar or larger project. Python is indeed a very handy language to write quick prototypes and proof of concepts, but issues become more frequent in larger projects.
