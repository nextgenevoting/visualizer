\chapter*{Management Summary}
The following documentation describes the context, planning and implementation of an application intended to visualize a new e-voting protocol. The protocol our application is based on is described in the paper "{}CHVote System Specification"{} of Rolf Haenni, Reto E. Koenig, Philipp Locher and Eric Dubuis of the Research Institute for Security in the Information Society (RISIS) of the Bern University of Applied Sciences. Their specifications describe how to build a next generation e-voting protocol which satisfies the complex requirements set up by the Swiss government, such as allowing an e-voting system to include large electorates.

Previous attempts to establish e-voting platforms in Switzerland were limited to only a small percentage of the electorate, for example Swiss citizens living abroad, because they only met the basic requirements, set up by the government. There exist many concepts and e-voting protocols which satisfy basic requirements, such as the privacy of the voters. However, an e-voting platform that could be used on a nationwide scale needs to be both individually and universally verifiable. This means in essence that an external individual can verify that all, but only valid votes have been counted in the tally. Current systems were behaving more like a black box and were not transparent enough to allow that kind of verification.

Another challenge which e-voting is facing is the educational problem: it is difficult to understand such a complex protocol without sufficient knowledge of cryptography. This might result in mistrust towards e-voting.

The goal of this project is to develop an application that addresses the educational problem of e-voting by allowing users to get a hands-on experience with the whole voting process from the perspective of each participating actor of the protocol. This would allow users to gain a better understanding of the next generation e-voting platform. The system should also be able to display multiple perspectives on different screens which requires real-time synchronization of data.

This document will first introduce the context of e-voting and our project task. Next it will describe the planning aspects such as the requirements and the time schedule. A short summary of the most important aspects of the CHVote protocol should establish the terminology and background knowledge for better understanding of our work. The main goal of this paper is to document in detail the implementation of our application, such as the architecture and the challenges we were facing. The document concludes with a short summary and reflection.
