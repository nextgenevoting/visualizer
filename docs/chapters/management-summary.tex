\chapter*{Management Summary}
The following documentation describes the context, planing and implementation of an application, intended to visualize a new e-voting protocol. The protocol our application is based on, is the "`CHVote System Specification"' of Rolf Haenni, Reto E. Koenig, Philipp Locher and Eric Dubuis of the Research Institute for Security in the Information Society (RISIS) of the Bern University of Applied Sciences. Their specifications describes how to build a next generation e-voting protocol, which satisfies the complex requirements set up by the Swiss government such that an e-voting system is allowed to include large electorates.

Previous attempts to establish e-voting platforms in Switzerland were limited to only small percentage of the electorate, for example Swiss citizens living abroad, because they didn't meet the many requirements set up by the government. There exist many concepts and e-voting protocols, which satisfy basic requirements such as the privacy of the voters. However, an e-voting platform that could be used on a nationwide scale, needs to be both individually and universally verifiable. This means in essence that an external individual can verify, that all, but only valid votes have been counted in the tally. Current systems were behaving more like a black box and were not transparent enough.

One challenge which e-voting is facing is the educational problem: it is difficult to understand such a complex protocol without sufficient knowledge of cryptography. This might result in mistrust to e-voting.

The goal of this project is to develop an application that addresses the educational problem of e-voting. Users should be able to get a hands-on experience with the voting process from the perspective of each participating actor of the protocol. This would allow a better understanding of the next generation e-voting platform. The system should also be able to display multiple perspectives on different screens which requires real-time synchronization of data.

This document will first introduce the reader to the context of e-voting and our project task. Next it will describe the planing aspects such as the requirements, the time schedule and a short summary of the most important aspects of the CHVote protocol should establish the terminology and background knowledge for better understanding our work. The main goal of this document is to document in details the implementation of our application, such as the architecture and the challenges we were facing. The document will be closed with a short summary and reflection.
