\chapter{Introduction}
\section{Electronic voting}
Since 2015, it is possible for Swiss citizens registered in the cantons Geneva and Neuchâtel and living abroad, to vote electronically. However, these systems did not yet meet the requirements in terms of security and transparency, to be accepted as a secure E-Voting platform on a nationwide scale.

One of the requirements that is hardest to achieve, is that the system must ensure the voters privacy while at the same time, it must be verifiable that only valid votes have been counted.

A contract was formed between the state of Geneva and the Bern University of Applied Sciences to work out a new protocol which does meet the complex requirements set up by the government. In 2017, the resulting specification document written by Haenni Rolf, Philipp Locher and Reto E. Koenig has been officially published and a proof-of-concept / prototype has been successfully implemented by the State of Geneva. 

\section{Project task}

Understanding such a complex protocol isn't easy and might be the reason why many people still do not trust electronic e-voting systems. In close consultation with the authors of the CHVote specification, we agreed to develop an application that allows users to get a hands-on experience with the CHVote e-voting system and makes it possible to show to an audience how the future of voting in Switzerland might possibly look like.

For this reason, we have implemented the protocol according to the specification and developed a web-based application on top of it, which allows to perform every step of an election, from generating the electorate data, to casting and confirming ballots from a voters point of view, to the post-election processes like mixing, decryption and tallying.
