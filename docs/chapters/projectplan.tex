\newcommand\tabularhead[1]{
\begin{table}[h]
  \caption{Use Case <<#1>>}
  \begin{tabular}{|p{0.3\linewidth}|p{0.7\linewidth}|}
    \hline
    \textbf{Use Case} & \textbf{#1} \\
    \hline}

  \newcommand\addrow[2]{#1 &#2\\ \hline}

  \newcommand\addmulrow[2]{ \begin{minipage}[t][][t]{2.5cm}#1\end{minipage}% 
     &\begin{minipage}[t][][t]{8cm}
      \begin{enumerate} #2   \end{enumerate}
      \end{minipage}\\ }

  \newenvironment{usecase}{\tabularhead}
{\hline\end{tabular}\end{table}}

\chapter{Project Plan}
\section{Project Method}
\section{Requirements}
\subsection{General Requirements}
\begin{longtable}{p{0.5cm}p{9cm}p{1cm}p{1cm}p{1cm}p{1cm}}
\hline
 & Description & Type & Prio. & Phase & Status\\
\hline
R1 & The CHVote protocol is implemented as specified in the latest specification document. The only exclusion are the algorithms for channel security. & Must & High & 1 & Done\\
R2 & The application is web-based shows updates within the same demo-election in real-time. & Must & High & 1 & Done\\
R3 & The system supports 1-out-of-3 type of elections (e.g. elect 1 of 3 possible candidates) & Must & High & 1 & Done\\
R4 & The system supports multiple parallel elections & Must & High & 1 & \\
R5 & Users can create new elections & Must & High & 1 & Done \\
R6 & The system supports internationalization. Providing more than one language is not required. & Must & Med. & 1 & \\
R7 & The system can handle k-out-of-n type of elections & Can & Med. & 1 & \\
\end{longtable}


\subsection{Election-Overview}
\begin{longtable}{p{0.5cm}p{9cm}p{1cm}p{1cm}p{1cm}p{1cm}}
\hline
 & Description & Type & Prio. & Phase & Status\\
\hline
R8 & The overview shows which phase the election is currently in & Must & High & 2 & \\
R9 & A graphical scheme of the chVote protocol gives an overview of all participating parties & Must & Med. & 2 & \\
\end{longtable}

\subsection{Election Administrator}
\begin{longtable}{p{0.5cm}p{9cm}p{1cm}p{1cm}p{1cm}p{1cm}}
\hline
 & Description & Type & Prio. & Phase & Status\\
\hline
R10 & An election can be set up by providing all required information such as the candidates, number of parallel voters, the number of voters and the number of selections (simplified JSON input) & Must & High & 1 & Done\\
R11 & The election can be set up without entering the parameters in JSON format and allows easier set up of elections with multiple parallel election events & Must & Low & 2 & \\
R12 & The election administrator view allows to perform the tallying and displays the final result of an election in numbers and a pie chart & Must & High & 1 & \\
R13 &  During election setup, the security parameters can be chosen from a set of predefined parameters & Can & Low & 2 & \\
\end{longtable}

\subsection{Printing Authority}
\begin{longtable}{p{0.5cm}p{9cm}p{1cm}p{1cm}p{1cm}p{1cm}}
\hline
 & Description & Type & Prio. & Phase & Status\\
\hline
R14 & Users can generate and display voting cards for an election. & Must & High & 1 & Done\\
R15 & Voting cards hide sensitive information behind a scratch card & Can & Med. & 2 & \\
\end{longtable}

\subsection{Election Authority}
\begin{longtable}{p{0.5cm}p{9cm}p{1cm}p{1cm}p{1cm}p{1cm}}
\hline
 & Description & Type & Prio. & Phase & Status\\
\hline
R16 & The election authority view shows all information known to an election authority & Must & High & 1 & Done\\
R17 & After a voter has submitted a ballot, all election authorities can check and respond to the voters submission & Must & High & 1 & Done\\
R18 & In the post-election phase, all election authorities can perform the mixing and decryption tasks & Must & High & 2 & \\
R19 & Each authority can optionally processes all tasks automatically & Can & High & 2 & Partially done\\
\end{longtable}


\subsection{Voter}
\begin{longtable}{p{0.5cm}p{9cm}p{1cm}p{1cm}p{1cm}p{1cm}}
\hline
 & Description & Type & Prio. & Phase & Status\\
\hline
R20 & Users are able to go through the whole vote-casting process for every voter & Must & High & 1 & Done\\
R21 & The voting card of a voter is displayed on screen. The voting and confirmation codes can be copied into the input textfields by double clicking & Must & Med. & 1 & \\
\end{longtable}


\subsection{Bulletin Board}
\begin{longtable}{p{0.5cm}p{9cm}p{1cm}p{1cm}p{1cm}p{1cm}}
\hline
 & Description & Type & Prio. & Phase & Status\\
\hline
R22 & The bulletin board view shows what information is publicly available & Must & High & 1 & Done\\
R23 & The bulletin board view is extended with verification-functionality & Can & Low & 2 & \\
\end{longtable}


\section{Timeplan}
We divided 

\section{Use Cases}
\begin{usecase}{Create new election}
  \addrow{Primary Actor}{User}
  \addrow{Description}{The system allows to create new elections}
    \addrow{Precondition}{The system shows the available elections in a list}

    \addmulrow{Main path (M)}{
        \item User clicks on "`create election"' 
        \item System demands a name for the election 
				\item User is redirected to the election overview page
				}				
\end{usecase}

\begin{usecase}{Set up election}
  \addrow{Primary Actor}{Election Administrator}
  \addrow{Description}{The election administrator can set up the election. This involves the generation of the cryptographic electorate data in the backend}
    \addrow{Precondition}{An new election has been created}
    \addrow{Postcondition}{The election has the status "`Printing"'}		
    \addmulrow{Main path (M)}{
        \item Election Administrator visits the "`Election Administrator"'-view of a new election. 
        \item The system demands the following information: 
				
				\begin{itemize}					
					\item Number of parallel elections
					\item Candidates per election
					\item Number of possible selections per election event
					\item Number of voters
					\item Counting circles of the voters
				\end{itemize}	
				\item User clicks on "`Generate"' \\
				}							
\end{usecase}

\begin{usecase}{Printing of voting cards}
  \addrow{Primary Actor}{Printing Authority}
  \addrow{Description}{The printing authority generates voting cards}
    \addrow{Precondition}{The election has the status "`Printing"'}
    \addrow{Postcondition}{The election has the status "`Delivery"'}		
    \addmulrow{Main path (M)}{
        \item The election administrator visits the "`Printing Authority"'-view of an election. 
        \item The election administrator clicks on "`Print Voting Cards"' 
				\item A list of all voters is displayed 
			  \item The election administrator can select a voter to see his voting card
				}							
\end{usecase}


\begin{usecase}{Delivery of voting cards}
  \addrow{Primary Actor}{Printing Authority}
  \addrow{Description}{The printing authority can send the voting cards to the voters}
    \addrow{Precondition}{The election has the status "`Delivery"'}
    \addrow{Postcondition}{The election has the status "`Election Phase"'}		
    \addmulrow{Main path (M)}{
        \item The election administrator visits the "`Printing Authority"'-view of an election. 
        \item The election administrator clicks on "`Deliver Voting Cards"' 
				\item Within the voters-view, the voting card shows up for every voter
				}							
\end{usecase}

\begin{usecase}{Casting of a vote}
  \addrow{Primary Actor}{Voter}
  \addrow{Description}{The voter can cast a vote by selecting his favored candidate(s) and his voting code}
    \addrow{Precondition}{
		
		\begin{itemize}
			\item The election has the status "`Election Phase"'
			\item A voter is selected in the "`Voter"'-view
			\item The voter has the status "`Vote Casting Phase"'
		\end{itemize}		}
    \addrow{Postcondition}{The first election authority receives a "`Ballot-check task"'}		
    \addmulrow{Main path (M)}{
        \item The voter visits the "`Voter"'-view and select his voter object
				\item The system demands a selection of the candidates and the voter's voting code
				\item The voter clicks on "`Cast ballot"'

				}							
\end{usecase}

\begin{usecase}{Confirmation of a vote}
  \addrow{Primary Actor}{Voter}
  \addrow{Description}{The voter can confirm his vote by verifying the verification codes and entering his confirmation code}
    \addrow{Precondition}{
		
		\begin{itemize}
			\item The election has the status "`Election Phase"'
			\item A voter is selected in the "`Voter"'-view
			\item The voter has the status "`Confirmation Phase"'
		\end{itemize}		}
    \addrow{Postcondition}{	
			The first election authority receives a "`Check-confirmation task"'	
		}		
    \addmulrow{Main path (M)}{
        \item The voter visits the "`Voter"'-view and select his voter object
				\item The system displays the verification codes of the selected candidates
				\item The voter must manually verify that the displayed codes match the verification codes of the selected candidates on his voting card
				\item The system demands the confirmation code
				\item The voter clicks on "`Confirm vote"'

				}							
\end{usecase}

\begin{usecase}{Checking a ballot}
  \addrow{Primary Actor}{Election Authority}
  \addrow{Description}{The election authority can verify the validity of a ballot and respond to the voters query}
    \addrow{Precondition}{
		\begin{itemize}
			\item The election has the status "`Election Phase"'
			\item The currently selected election authority has a new "`Check ballot task"'
		\end{itemize}		}
    \addrow{Postcondition}{
		
		\begin{itemize}
			\item The next election authority receives a "`Check ballot task"'
			\item If this election authority was the last one, and the ballot was valid, the voter now has the status "`Confirmation Phase"'
		\end{itemize}
		}		
    \addmulrow{Main path (M)}{
        \item The user visits the "`Election Authority"'-view and select one of the available election authorities that has new "`Check ballot task"'
				\item The system displays the query, the ballot proof and the voting credential of the voter
				\item The user click on "`Check validity"'
				\item The system displays the result of the validity check
				\item The user clicks on "`Respond"'
				}							
\end{usecase}

\begin{usecase}{Checking a confirmation}
  \addrow{Primary Actor}{Election Authority}
  \addrow{Description}{The election authority can verify the validity of a confirmation and respond to the voters query}
    \addrow{Precondition}{
		\begin{itemize}
			\item The election has the status "`Election Phase"'
			\item The currently selected election authority has a new "`Check ballot task"'
		\end{itemize}		}
    \addrow{Postcondition}{
		
		\begin{itemize}
			\item The next election authority receives a "`Check confirmation task"'
			\item If this election authority was the last one, and the confirmation was valid, the voter now has the status "`Finalization Phase"'
		\end{itemize}
		}		
    \addmulrow{Main path (M)}{
        \item The user visits the "`Election Authority"'-view and select one of the available election authorities that has new "`Check confirmation task"'
				\item The system displays information about the confirmation
				\item The user click on "`Check validity"'
				\item The system displays the result of the validity check
				\item The user clicks on "`Finalize"'
				}							
\end{usecase}

\begin{usecase}{Mixing}
  \addrow{Primary Actor}{Election Authority}
  \addrow{Description}{Every election authority can perform the mixing on the extracted list of encryptions}
    \addrow{Precondition}{
		\begin{itemize}
			\item The election has the status "`Mixing"'
			\item The previous election authority has already performed the mixing
		\end{itemize}		}
    \addrow{Postcondition}{
		
		\begin{itemize}
			\item The next election authority is able to mix
		\end{itemize}
		}		
    \addmulrow{Main path (M)}{
        \item The user visits the "`Election Authority"'-view and select one of the available election authorities that hasn't mixed before
				\item The system displays the list of encryptions of the previous election authority (or the first one in case the first election authority is selected)
				\item The user clicks on "`Mix"'
				\item The new, mixed list of encryptions is added to the known data of this election authority
				}							
\end{usecase}


\begin{usecase}{Decryption}
  \addrow{Primary Actor}{Election Authority}
  \addrow{Description}{Every election authority can perform the (partial) decryption}
    \addrow{Precondition}{
		\begin{itemize}
			\item The election has the status "`Decryption"'
			\item The previous election authority has already performed the decryption
		\end{itemize}		}
    \addrow{Postcondition}{
		\begin{itemize}
			\item The next election authority is able to decrypt
		\end{itemize}
		}		
    \addmulrow{Main path (M)}{
        \item The user visits the "`Election Authority"'-view and select one of the available election authorities that hasn't decrypted before
				\item The system displays the list of encryptions
				\item The user clicks on "`Decrypt"'
				\item The list of partial decryptions is added to the known data of this election authority
				}							
\end{usecase}


\begin{usecase}{Tallying}
  \addrow{Primary Actor}{Election Administrator}
  \addrow{Description}{The election administrator can perform the tallying and view the final result}
    \addrow{Precondition}{
		The election has the status "`Tallying"'
		}
    \addrow{Postcondition}{
		The has the status "`Finished"'
		}		
    \addmulrow{Main path (M)}{
        \item The user visits the "`Election Administrator"'-view
				\item The user clicks on "`Tally"'
				\item The final result is added to the known data of the election administrator
				}							
\end{usecase}

\section{Concept}
TODO: Beschreiben, wie wir die Anwendung aus fachlicher Sicht aufteilen wollen (Unterteilung in eine View per Actor etc.)

\section{Layout}
TODO: Beschreiben wie wir die Anwendung optisch/layoutmässig geplant haben
