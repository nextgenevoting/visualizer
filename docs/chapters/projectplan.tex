\chapter{Project Plan}
\section{Project Method}
\section{Requirements}
\subsection{General Requirements}
\begin{longtable}{p{0.4cm}p{9cm}p{1cm}p{1cm}p{1cm}p{1cm}}
\hline
 & Description & Must & Priority & Iteration & Status\\
\hline
R1 & The CHVote protocol is implemented as specified in the latest specification document. The only exclusion are the algorithms for channel security. & Must & High & 1 & Done\\
R2 & The application is web-based shows updates within the same demo-election in real-time. & Must & High & 1 & Done\\
R3 & The system supports 1-out-of-3 type of elections (e.g. elect 1 of 3 possible candidates) & Must & High & 1 & Done\\
R4 & The system supports multiple parallel elections & Must & High & 1 & \\
R5 & The system supports internationalization. Providing more than one language is not required. & Must & High & 1 & Part. done\\
R6 & Users can create new elections & Must & High & 1 & Done \\
R7 & The system can handle k-out-of-n type of elections & Can & Medium & 1 & \\
\end{longtable}


\subsection{Election-Overview}
\begin{longtable}{p{0.4cm}p{9cm}p{1cm}p{1cm}p{1cm}p{1cm}}
\hline
 & Description & Must & Priority & Iteration & Status\\
\hline
R8 & The overview shows which phase the election is currently in & Must & High & 2 & \\
R9 & A graphical scheme of the chVote protocol gives an overview of all participating parties & Must & Medium & 2 & \\
\end{longtable}

\subsection{Election Administrator}
\begin{longtable}{p{0.4cm}p{9cm}p{1cm}p{1cm}p{1cm}p{1cm}}
\hline
 & Description & Must & Priority & Iteration & Status\\
\hline
R10 & An election can be set up by providing all required information such as the candidates, number of parallel voters, the number of voters and the number of selections (simplified JSON input) & Must & High & 1 & Done\\
R11 & The election can be set up without entering the parameters in JSON format and allows easier set up of elections with multiple parallel election events & Must & Low & 2 & \\
R12 & The election administrator view allows to perform the tallying and displays the final result of an election in numbers and a pie chart & Must & High & 1 & \\
R13 &  During election setup, the security parameters can be chosen from a set of predefined parameters & Can & Low & 2 & \\
\end{longtable}

\subsection{Printing Authority}
\begin{longtable}{p{0.4cm}p{9cm}p{1cm}p{1cm}p{1cm}p{1cm}}
\hline
 & Description & Must & Priority & Iteration & Status\\
\hline
R14 & Users can generate and display voting cards for an election. & Must & High & 1 & Done\\
R15 & Voting cards hide sensitive information behind a scratch card & Can & Medium & 2 & \\
\end{longtable}

\subsection{Election Authority}
\begin{longtable}{p{0.4cm}p{9cm}p{1cm}p{1cm}p{1cm}p{1cm}}
\hline
 & Description & Must & Priority & Iteration & Status\\
\hline
R16 & The election authority view shows all information known to an election authority & Must & High & 1 & Done\\
R17 & After a voter has submitted a ballot, all election authorities can check and respond to the voters submission & Must & High & 1 & Done\\
R18 & In the post-election phase, all election authorities can perform the mixing and decryption tasks & Must & High & 2 & \\
R19 & Each authority can optionally processes all tasks automatically & Can & High & 2 & Partially done\\
\end{longtable}


\subsection{Voter}
\begin{longtable}{p{0.4cm}p{9cm}p{1cm}p{1cm}p{1cm}p{1cm}}
\hline
 & Description & Must & Priority & Iteration & Status\\
\hline
R20 & Users are able to go through the whole vote-casting process for every voter & Must & High & 1 & Done\\
R21 & The voting card of a voter is displayed on screen. The voting and confirmation codes can be copied into the input textfields by double clicking & Must & Medium & 1 & \\
\end{longtable}


\subsection{Bulletin Board}
\begin{longtable}{p{0.4cm}p{9cm}p{1cm}p{1cm}p{1cm}p{1cm}}
\hline
 & Description & Must & Priority & Iteration & Status\\
\hline
R22 & The bulletin board view shows what information is publicly available & Must & High & 1 & Done\\
R23 & The bulletin board view is extended with verification-functionality & Can & Low & 2 & \\
\end{longtable}


\section{Timeplan}
