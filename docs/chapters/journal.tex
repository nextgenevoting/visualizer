\chapter{Journal}

\section{Week 1}
\subsection{Kickoff Meeting}
During our kickoff meeting we discussed the possibilities of our bachelor thesis based on the spadework of the previous "project 2" module and broke them down into two options: A realistic prototype of the whole CHVote system which includes everything a real implementation would need, like signatures, channel security, distributed election authorities with docker etc., or to build a demonstrator tool (an application that allows to demonstrate the functionality of the chVote protocol in a more visual manner).

\section{Week 2}
In week 2 we have made the decision to build the demonstrator mainly because the final product would potentially be more attractive visually than a prototype where the main work lies in the background which is not visible to an outsider. We have also started thinking about what technologies and frameworks to use and to build a few sketches and mockups to have some basis for discussion for our next meeting.

We have also started updating our implementation of the CHVote crypto-library to the latest specification. Since we ran into a few problems, this took us almost the whole week.

\subsection{Reflexion}
At this stage we have yet been very unsure about how the application should look like, what audience we should have as our main target and what functionality the application should offer.

\section{Week 3}
During our second meeting we discussed the further elaborated the goals, the structure of our project and talked about the audience.
\begin{itemize}
\item In essence, the application should allow to demonstrate a chVote election from the view of every party participating in the election process.
\item The application should be a real-time webapp that updates the views automatically as soon as something changes and without having to reload the page
\end{itemize}

\section{Week 4}
In the fourth week we continued describing the goals and further worked on the system architecture. We also made some first experiences with the envisaged frameworks and technologies (VueJS, socket.io, Flask, MongoDB).
\subsection{Reflexion}
\begin{itemize}
\item working with socket.io and VueJS has been very intuitive and looked very promising and suitable for our project
\item We were not yet sure whether or not mongoDB is the right technology for our needs.
\end{itemize}
During prototyping, we observed that our first architecture approach of doing everything over websockets, turned out to be a bad decision. 

\section{Week 5}
In the fifth week we started with the real implementation.
\subsection{Reflexion}
\begin{itemize}
\item working with socket.io and VueJS has been very intuitive and looked very promising and suitable for our project
\item We were not yet sure whether or not mongoDB is the right technology for our needs.
\end{itemize}
During prototyping, we observed that our first architecture approach of doing everything over websockets, turned out to be a bad decision. 

